\section{Assignment 2: Basic Probabilities and Visualizations}

\subsection{$\xi$ values}

\begin{itemize}
    \item $\xi_9 = 0$
    \item $\xi_10 = 8,7,6,17,12$
\end{itemize}



\section{Problem statement}
The total bandwidth to failure \( S \) of a single router follows an exponential distribution with density:
\[
f_S(s) = \frac{1}{\theta} \exp\left(-\frac{s}{\theta}\right), \quad s > 0, \, \theta > 0,
\]
where \( \theta \) is the mean failure bandwidth for a single router.

For a dual-router system, the total bandwidth to failure \( T \) can be expressed as:
\[
T = S_1 + S_2,
\]
where \( S_1 \) and \( S_2 \) are independent and identically distributed random variables representing the bandwidth totals to failure of each router.

\section{Density Function of \( T \)}
Given \( S_1 \sim \text{Exp}(\theta) \) and \( S_2 \sim \text{Exp}(\theta) \), the sum \( T = S_1 + S_2 \) follows a \emph{Gamma distribution} with shape parameter \( k = 2 \) and rate \( \lambda = \frac{1}{\theta} \). The probability density function of \( T \) is:
\[
f_T(t) = \frac{t^{k-1} \lambda^k e^{-\lambda t}}{\Gamma(k)}, \quad t > 0,
\]
where \( \Gamma(2) = 1 \). Substituting \( k = 2 \) and \( \lambda = \frac{1}{\theta} \), we get:
\[
f_T(t) = \frac{t \lambda^2 e^{-\lambda t}}{1} = \frac{t}{\theta^2} \exp\left(-\frac{t}{\theta}\right), \quad t > 0.
\]

\section{Likelihood Function}
Given an independent sample \( T_1, T_2, \dots, T_n \) of \( T \), the likelihood function for the parameter \( \theta \) is:
\[
L(\theta) = \prod_{i=1}^n f_T(T_i) = \prod_{i=1}^n \frac{T_i}{\theta^2} \exp\left(-\frac{T_i}{\theta}\right).
\]
Simplifying:
\[
L(\theta) = \frac{1}{\theta^{2n}} \prod_{i=1}^n T_i \exp\left(-\frac{1}{\theta} \sum_{i=1}^n T_i\right).
\]

\section{Simplification of the Likelihood Function}
To maximize the likelihood function, we simplify using the log-likelihood:
\[
\ell(\theta) = \ln L(\theta) = -2n \ln \theta + \sum_{i=1}^n \ln T_i - \frac{1}{\theta} \sum_{i=1}^n T_i.
\]
The derivative of \( \ell(\theta) \) with respect to \( \theta \) is:
\[
\frac{\partial \ell}{\partial \theta} = -\frac{2n}{\theta} + \frac{1}{\theta^2} \sum_{i=1}^n T_i.
\]
Setting this equal to zero gives:
\[
\hat{\theta} = \frac{1}{2n} \sum_{i=1}^n T_i.
\]

\section{Estimation and Expectation for the Experiment}
Given the sample \( [8, 7, 6, 17, 12] \), we compute \( \hat{\theta} \) and the expectation of \( T \) as:
\[
\mathbb{E}[T] = 2\hat{\theta}.
\]
